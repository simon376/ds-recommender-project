\documentclass[10pt,draft,journal,a4paper,oneside,twocolumn]{IEEEtran}
\IEEEoverridecommandlockouts
% The preceding line is only needed to identify funding in the first footnote. If that is unneeded, please comment it out.
\usepackage{hyperref}
\usepackage{cite}
\usepackage{amsmath,amssymb,amsfonts}
\usepackage{algorithmic}
\usepackage{graphicx}
\usepackage{textcomp}
\usepackage{xcolor}
\usepackage{tabularx}
\usepackage{booktabs}
\def\BibTeX{{\rm B\kern-.05em{\sc i\kern-.025em b}\kern-.08em
    T\kern-.1667em\lower.7ex\hbox{E}\kern-.125emX}}
\usepackage{todonotes}
\begin{document}

\title{Exploring Deep-Learning Recommender Systems for Book Recommendations}

\author{\IEEEauthorblockN{1\textsuperscript{st} Simon Müller}
\IEEEauthorblockA{\textit{Computer Science student, Data Science class} \\
\textit{University of Applied Sciences Augsburg}\\
Augsburg, Germany \\
simon.mueller3@hs-augsburg.de}}

\maketitle

\begin{abstract}
This document is a model and instructions for \LaTeX.
This and the IEEEtran.cls file define the components of your paper [title, text, heads, etc.]. *CRITICAL: Do Not Use Symbols, Special Characters, Footnotes, 
or Math in Paper Title or Abstract.
\end{abstract}

\begin{IEEEkeywords}
component, formatting, style, styling, insert
\end{IEEEkeywords}

\section{Introduction}
The current time has been described not only as the "information age", but also as an attention economy. Corporations compete for the limited attention of their prospective customers. On social media platforms, users have to be made to spend as much time as possible on the services so as much data on them as possible can be gathered and more advertisements can be shown. Subscription services want to present their users with perfectly fitting content to keep them interested in their platform and subscribed to their service. Online warehouses have an even more direct interest in providing the best fitting products to customers to directly influence their sales.

Recommendation systems are used especially in the advertising and sales business domains. There, they are used to predict click-through rates on ads, which is the probability of a user interacting with a displayed online advertisement. This is of utmost importance for advertisement providers like Alphabet or Meta, which can increase their advertisement prices based on how well they know their potential advertisement target audience. 

For all of these examples, the users of the respective platforms and services have to be recommended the most appropriate products, pieces of content, or advertisements to keep the user's interest in the service. This task is the goal of the recommender or recommendation systems, which will be explored in this paper on a concrete data set.

\section{state of the art}
As described in the Introduction, most interest in the topic arises in the advertisement and sales business, which is why a lot of research is being done by the research departments of Google/Alphabet, Meta (formerly Facebook), Microsoft, and Amazon.

In the following section, a general ontology of recommendation systems will be presented.
There are two main types of recommendation methods, content-based and Collaborative filtering. They differ in their main goals and the data available. 
Collaborative Filtering on one hand tries not to perfectly know the specific item the targeted user may be most interested in but instead tries to best categorize the user base into different groups with similar interests. This is done because users with similar general interests generally respond the same to specific recommended items, and often a "perfect fit" is unnecessary or virtually impossible to achieve because of the sparsity of the available data.
Collaborative Filtering methods are especially useful for predicting click-through rates for advertisements, because in this context, usually only implicit feedback is available. Implicit feedback is feedback data tracked by cookies with only relatively insignificant correlations. A click on a search result, on an item on a shopping website, a like to a social media post are considered implicit feedback.  When the data consists of millions of these small interactions, this impliicit feedback may be used to give explicit recommendations. For this approach, a huge data set is needed, since a single click is of little value, since i.e. it may be a misclick without any relevance to the user. Implicit feedback dataset matrices usually consists of thousands of features, which are occupied only very sparsely.

\todo{und wie wird das berechnet? formel}


Content-based filtering on the other hand instead tries to achieve the opposite. These algorithms take as much information available about the items (the content) as possible to predict the interest of a user for it, or the likelihood of an interaction respectively. These approaches usually rely on explicit feedback by the users, which make predictions more predictable. They are used in domains like streaming services, where for example the service provider can estimate the user liked a movie if he watched in completely, and especially if the user gives an explicit item rating, usually on a 1-5 stars or at least a like/dislike basis.

\todo{berechnung}
% matrix factorization

A basic approach to predict these likelihoods is to estimate a user-item-matrix, with each unique user as a row and all the available items as columns, where the matrix cells [i,j] are filled with a calculated probability of interaction or of the predicated rating / interest. 

Deep-Learning-based sytems try to combine these approaches while letting a deep neural network architecture take care of the feature selection to gain the best recommendations.

\todo{Einführung in RecSys - Collaborative vs Content-based Filtering, Matrix Factorization (ok das nicht). }
\todo{seit allerneustem auch in pytorch}
\section{related work}
DLRM, DCN, NCF, DeepFM, ....
same same, but different?
\section{research question}
naja.. gibts ned. will halt bekannten algorithmus auf neuen Datensatz anwenden.. und mehrere vergleichen
Ziel: Einfluss von side features (text) auf recommendations, statt nur interactions
\section{concept}
...
auswahl des models / der models
\section{realisation/evaluation}
Tensorflow was used as the machine learning framework to implement both the data processing pipeline as well as the  
deep learning recommendation system models.

etl pipeline

\section{conclustion/outlook}
\section{data set}
The data set used for this work, is the UCSD Book Graph \cite{todo}, created by *authors* of the University of California, San Diego (UCSD). The dataset was scraped in late 2017 from the website goodreads.com and last updated in 2019 and consists only of publicly available data. The provided dataset is of enormous size, containing information about more than two million books, more than 800k users, with more than 200 million interactions between the latter.
\todo{was ist goodreads, was sind shelves / interactions}
Because of the sheer scale of the data set, the authors recommend using only a subset of the data, especially when used on a local machine. For this, they have created subsets for different genres, like "Children", "History \& Biography" or "Mystery, Thriller \& Crime". These categorizations may not be perfect and may overlap, since a genre specification is not directly available in the book metadata, but is instead inferred from the shelves users have put the books on.
\paragraph{Books}
The book data set contains 2.36 million entries, including information like a unique ID, title, author, release date and a description text.
The available columns are shown in \autoref{fig:book_cols}. 
\begin{figure}[h]
    \centering
    \missingfigure{tabelle book columns}
    \caption{Some caption}
    \label{fig:book_cols}
\end{figure}

\begin{table}
    
    \begin{tabular}{clll}
        \toprule
    \# & Column & Non-Null Count & Dtype \\
    \midrule
    0 & title & 219235 non-null & string \\
    1 & text\_reviews\_count & 219235 non-null & uint32 \\
    2 & average\_rating & 219235 non-null & float64 \\
    3 & description & 198488 non-null & string \\
    4 & author\_id & 219235 non-null & int64 \\
    \bottomrule
\end{tabular}
\end{table}
    

\paragraph{Shelves}
Users of the goodreads website may categorize their books into different shelves. Some of them may be exclusive. The most prominent shelves are the three pre-defined shelves "to-read", "read", and "want-to-read". On top of that, users may add as many shelves as they like and place books into one or more of them to keep track of their book collection.
The shelf-interaction dataset may be used as implicit feedback, as described in \autoref{stateoftheart}. As stated previously, the dataset is very large and sparse. A user having shelfed a book does not necessarily imply strong interest in the book, but it is in indicator. 
The data consists of mappings between user ID and book ID, with additional indicators if the book has been read (inferred from the existence of the book in the "read" shelf), the user's rating if available and if the user created a review on it. These additional indicators may be used as explicit feedback.
A more detailed data set adds time information to this, recording when the user has started and finished the book. The authors of the dataset themselves have used this data to create a recommender system using time series data \cite{Pera.2018}.

\paragraph{Reviews}
On top of the book metadata and basic user interactions with these books, textual reviews have been scraped. The complete dataset contains multilingual review text without spoiler tags. It contains more than 15M reviews about ~2M books and 465K users. Users of the website can add spoiler tags to hide sensitive plot information from other readers, these tags have been removed in the dataset. This data was used to train a neural network to autonomously detect spoilers in text data \cite{Wan.2019}.

The columns of this data are shown in \autoref{fig:review_cols}
\begin{table*}
\centering
\caption{Sample of the Review Data}
\label{tab:review_cols}
\begin{tabularx}{\textwidth}{llrrlllllrr}
\toprule
{} &    user\_id &   book\_id &  rating &                                        review\_text &                      date\_added &                    date\_updated &                         read\_at &                      started\_at &  n\_votes &  n\_comments \\
review\_id &            &           &         &                                                    &                                 &                                 &                                 &                                 &          &             \\
\midrule
% e84598... &  a5ba0a... & 227043... &       4 &  I have been waiting for this book forever. Ple... &  Fri Aug 22 10:04:43 -0700 2014 &  Mon Oct 06 14:16:09 -0700 2014 &  Mon Oct 06 14:16:09 -0700 2014 &                                 &        1 &           0 \\
% e864fe... &  7e4945... & 181951... &       4 &  I enjoyed this book by Louise Phillips.I had a... &  Mon Sep 09 07:15:58 -0700 2013 &  Sun Jan 18 09:08:25 -0800 2015 &  Sun Jan 18 09:36:30 -0800 2015 &  Wed Dec 17 00:00:00 -0800 2014 &        1 &           0 \\
% d57341... &  7e4945... & 281872... &       4 &  I enjoyed this book by Ruth Ware.Set on a luxu... &  Wed Apr 27 08:01:27 -0700 2016 &  Thu Sep 08 06:35:57 -0700 2016 &  Thu Sep 08 06:35:57 -0700 2016 &  Sun Aug 21 00:00:00 -0700 2016 &        0 &           0 \\
% d4bab4... &  7e4945... & 174521... &       4 &  I really enjoyed this book, plenty of plot twi... &  Fri May 24 07:44:51 -0700 2013 &  Mon Apr 07 06:02:53 -0700 2014 &  Mon Apr 07 06:02:53 -0700 2014 &  Sat Mar 22 00:00:00 -0700 2014 &        0 &           0 \\
% 900706... &  187597... & 602950... &       5 &  Segun wikipedia la novela de misterio mas vend... &  Tue Aug 30 00:15:51 -0700 2016 &  Sun Sep 04 12:51:41 -0700 2016 &  Sun Sep 04 12:51:41 -0700 2016 &  Tue Aug 30 00:00:00 -0700 2016 &        3 &           4 \\
\bottomrule
\end{tabularx}
\end{table*}


\begin{figure}[h]
    \centering
    \missingfigure{tabelle book columns}
    \caption{Some caption}
    \label{fig:review_cols}
\end{figure}
wer hats gemacht, was ist enthalten, wie kodiert
explicit feedback: ratings
less explicit feedback: interactions
side features: review texts, descriptions, etc.

\paragraph{Data used in this work}
As described in the preceding paragraphs, the provided dataset is of an industrial scale. To explore the data and train neural networks in a reasonable time, only a subset of data is used.
The "Mystery, Thriller \& Crime" subset was chosen for this work. It consists of 219,235 books and 1,849,236 detailed text reviews. The interaction data was not used \todo{warum}.

\subsection{Data Processing}
The selected data needs to be prprocessed to be used in a deep learning framework.

\subsubsection{Data Download and JSON Processing}
* download from gdrive
* json to csv

\subsubsection{data preprocessing/cleaning in pandas}
* drop columns
* rename columns
* convert dtypes
* only using subset of datapoints: ~ 1 mio entries in total, X unique books, Y unique users

* merge dataset into one dataframe

* train test validation split (warum alle 3?)
* ttv size, batching, ...

\subsubsection{Feature Processing}
data of different types:
* continous numeric (review count, avg rating)
* discrete (alpha)numeric (book/author/user ids)
* diverse text features: (review text, title, description)

\paragraph{Encoding}
Strings \& Integers Encoded (mapped) using String/IntegerLookup 
\paragraph{Embeddings}
learning dense float (?) embeddings from those string / integer lookup tables

for non-discrete text data, preprocess the data usign TExtVectorization: split , lowercase, etc. then learn the embeddings.

normalize numeric data (review categorizes don't occur the same)

\section{Neural Network Model}
only one model in different configurations used: Deep\&Cross Network.
(describe network)
tfrs ranking task

(subsection?)
* Hyperparam tuning: HyperBand (Successive Halving)
* DNN only (no cross layers)
* show different parameters...

(Table: keras -> parameters -> to df -> to latex)


\section{Results}
plots vom hyperparamtuning, resultate, evtl. beispiel.

mit/ohne Cross layers
mit/ohne text data ( :( )

Embedding Visualization?

Appendix: EmailCallback


\subsection{Maintaining the Integrity of the Specifications}

The IEEEtran class file is used to format your paper and style the text. All margins, 
column widths, line spaces, and text fonts are prescribed; please do not 
alter them. You may note peculiarities. For example, the head margin
measures proportionately more than is customary. This measurement 
and others are deliberate, using specifications that anticipate your paper 
as one part of the entire proceedings, and not as an independent document. 
Please do not revise any of the current designations.

\section{Prepare Your Paper Before Styling}
Before you begin to format your paper, first write and save the content as a 
separate text file. Complete all content and organizational editing before 
formatting. Please note sections \ref{AA}--\ref{SCM} below for more information on 
proofreading, spelling and grammar.

Keep your text and graphic files separate until after the text has been 
formatted and styled. Do not number text heads---{\LaTeX} will do that 
for you.

\subsection{Abbreviations and Acronyms}\label{AA}
Define abbreviations and acronyms the first time they are used in the text, 
even after they have been defined in the abstract. Abbreviations such as 
IEEE, SI, MKS, CGS, ac, dc, and rms do not have to be defined. Do not use 
abbreviations in the title or heads unless they are unavoidable.

\subsection{Units}
\begin{itemize}
\item Use either SI (MKS) or CGS as primary units. (SI units are encouraged.) English units may be used as secondary units (in parentheses). An exception would be the use of English units as identifiers in trade, such as ``3.5-inch disk drive''.
\item Avoid combining SI and CGS units, such as current in amperes and magnetic field in oersteds. This often leads to confusion because equations do not balance dimensionally. If you must use mixed units, clearly state the units for each quantity that you use in an equation.
\item Do not mix complete spellings and abbreviations of units: ``Wb/m\textsuperscript{2}'' or ``webers per square meter'', not ``webers/m\textsuperscript{2}''. Spell out units when they appear in text: ``. . . a few henries'', not ``. . . a few H''.
\item Use a zero before decimal points: ``0.25'', not ``.25''. Use ``cm\textsuperscript{3}'', not ``cc''.)
\end{itemize}

\subsection{Equations}
Number equations consecutively. To make your 
equations more compact, you may use the solidus (~/~), the exp function, or 
appropriate exponents. Italicize Roman symbols for quantities and variables, 
but not Greek symbols. Use a long dash rather than a hyphen for a minus 
sign. Punctuate equations with commas or periods when they are part of a 
sentence, as in:
\begin{equation}
a+b=\gamma\label{eq}
\end{equation}

Be sure that the 
symbols in your equation have been defined before or immediately following 
the equation. Use ``\eqref{eq}'', not ``Eq.~\eqref{eq}'' or ``equation \eqref{eq}'', except at 
the beginning of a sentence: ``Equation \eqref{eq} is . . .''

\subsection{\LaTeX-Specific Advice}

Please use ``soft'' (e.g., \verb|\eqref{Eq}|) cross references instead
of ``hard'' references (e.g., \verb|(1)|). That will make it possible
to combine sections, add equations, or change the order of figures or
citations without having to go through the file line by line.

Please don't use the \verb|{eqnarray}| equation environment. Use
\verb|{align}| or \verb|{IEEEeqnarray}| instead. The \verb|{eqnarray}|
environment leaves unsightly spaces around relation symbols.

Please note that the \verb|{subequations}| environment in {\LaTeX}
will increment the main equation counter even when there are no
equation numbers displayed. If you forget that, you might write an
article in which the equation numbers skip from (17) to (20), causing
the copy editors to wonder if you've discovered a new method of
counting.

{\BibTeX} does not work by magic. It doesn't get the bibliographic
data from thin air but from .bib files. If you use {\BibTeX} to produce a
bibliography you must send the .bib files. 

{\LaTeX} can't read your mind. If you assign the same label to a
subsubsection and a table, you might find that Table I has been cross
referenced as Table IV-B3. 

{\LaTeX} does not have precognitive abilities. If you put a
\verb|\label| command before the command that updates the counter it's
supposed to be using, the label will pick up the last counter to be
cross referenced instead. In particular, a \verb|\label| command
should not go before the caption of a figure or a table.

Do not use \verb|\nonumber| inside the \verb|{array}| environment. It
will not stop equation numbers inside \verb|{array}| (there won't be
any anyway) and it might stop a wanted equation number in the
surrounding equation.

\subsection{Some Common Mistakes}\label{SCM}
\begin{itemize}
\item The word ``data'' is plural, not singular.
\item The subscript for the permeability of vacuum $\mu_{0}$, and other common scientific constants, is zero with subscript formatting, not a lowercase letter ``o''.
\item In American English, commas, semicolons, periods, question and exclamation marks are located within quotation marks only when a complete thought or name is cited, such as a title or full quotation. When quotation marks are used, instead of a bold or italic typeface, to highlight a word or phrase, punctuation should appear outside of the quotation marks. A parenthetical phrase or statement at the end of a sentence is punctuated outside of the closing parenthesis (like this). (A parenthetical sentence is punctuated within the parentheses.)
\item A graph within a graph is an ``inset'', not an ``insert''. The word alternatively is preferred to the word ``alternately'' (unless you really mean something that alternates).
\item Do not use the word ``essentially'' to mean ``approximately'' or ``effectively''.
\item In your paper title, if the words ``that uses'' can accurately replace the word ``using'', capitalize the ``u''; if not, keep using lower-cased.
\item Be aware of the different meanings of the homophones ``affect'' and ``effect'', ``complement'' and ``compliment'', ``discreet'' and ``discrete'', ``principal'' and ``principle''.
\item Do not confuse ``imply'' and ``infer''.
\item The prefix ``non'' is not a word; it should be joined to the word it modifies, usually without a hyphen.
\item There is no period after the ``et'' in the Latin abbreviation ``et al.''.
\item The abbreviation ``i.e.'' means ``that is'', and the abbreviation ``e.g.'' means ``for example''.
\end{itemize}
An excellent style manual for science writers is \cite{b7}.

\subsection{Authors and Affiliations}
\textbf{The class file is designed for, but not limited to, six authors.} A 
minimum of one author is required for all conference articles. Author names 
should be listed starting from left to right and then moving down to the 
next line. This is the author sequence that will be used in future citations 
and by indexing services. Names should not be listed in columns nor group by 
affiliation. Please keep your affiliations as succinct as possible (for 
example, do not differentiate among departments of the same organization).

\subsection{Identify the Headings}
Headings, or heads, are organizational devices that guide the reader through 
your paper. There are two types: component heads and text heads.

Component heads identify the different components of your paper and are not 
topically subordinate to each other. Examples include Acknowledgments and 
References and, for these, the correct style to use is ``Heading 5''. Use 
``figure caption'' for your Figure captions, and ``table head'' for your 
table title. Run-in heads, such as ``Abstract'', will require you to apply a 
style (in this case, italic) in addition to the style provided by the drop 
down menu to differentiate the head from the text.

Text heads organize the topics on a relational, hierarchical basis. For 
example, the paper title is the primary text head because all subsequent 
material relates and elaborates on this one topic. If there are two or more 
sub-topics, the next level head (uppercase Roman numerals) should be used 
and, conversely, if there are not at least two sub-topics, then no subheads 
should be introduced.

\subsection{Figures and Tables}
\paragraph{Positioning Figures and Tables} Place figures and tables at the top and 
bottom of columns. Avoid placing them in the middle of columns. Large 
figures and tables may span across both columns. Figure captions should be 
below the figures; table heads should appear above the tables. Insert 
figures and tables after they are cited in the text. Use the abbreviation 
``Fig.~\ref{fig}'', even at the beginning of a sentence.

\begin{table}[htbp]
\caption{Table Type Styles}
\begin{center}
\begin{tabular}{|c|c|c|c|}
\hline
\textbf{Table}&\multicolumn{3}{|c|}{\textbf{Table Column Head}} \\
\cline{2-4} 
\textbf{Head} & \textbf{\textit{Table column subhead}}& \textbf{\textit{Subhead}}& \textbf{\textit{Subhead}} \\
\hline
copy& More table copy$^{\mathrm{a}}$& &  \\
\hline
\multicolumn{4}{l}{$^{\mathrm{a}}$Sample of a Table footnote.}
\end{tabular}
\label{tab1}
\end{center}
\end{table}

\begin{figure}[htbp]
\centerline{\includegraphics{fig1.png}}
\caption{Example of a figure caption.}
\label{fig}
\end{figure}

Figure Labels: Use 8 point Times New Roman for Figure labels. Use words 
rather than symbols or abbreviations when writing Figure axis labels to 
avoid confusing the reader. As an example, write the quantity 
``Magnetization'', or ``Magnetization, M'', not just ``M''. If including 
units in the label, present them within parentheses. Do not label axes only 
with units. In the example, write ``Magnetization (A/m)'' or ``Magnetization 
\{A[m(1)]\}'', not just ``A/m''. Do not label axes with a ratio of 
quantities and units. For example, write ``Temperature (K)'', not 
``Temperature/K''.

\section*{Acknowledgment}

The preferred spelling of the word ``acknowledgment'' in America is without 
an ``e'' after the ``g''. Avoid the stilted expression ``one of us (R. B. 
G.) thanks $\ldots$''. Instead, try ``R. B. G. thanks$\ldots$''. Put sponsor 
acknowledgments in the unnumbered footnote on the first page.

\section*{References}

Please number citations consecutively within brackets \cite{b1}. The 
sentence punctuation follows the bracket \cite{b2}. Refer simply to the reference 
number, as in \cite{b3}---do not use ``Ref. \cite{b3}'' or ``reference \cite{b3}'' except at 
the beginning of a sentence: ``Reference \cite{b3} was the first $\ldots$''

Number footnotes separately in superscripts. Place the actual footnote at 
the bottom of the column in which it was cited. Do not put footnotes in the 
abstract or reference list. Use letters for table footnotes.

Unless there are six authors or more give all authors' names; do not use 
``et al.''. Papers that have not been published, even if they have been 
submitted for publication, should be cited as ``unpublished'' \cite{b4}. Papers 
that have been accepted for publication should be cited as ``in press'' \cite{b5}. 
Capitalize only the first word in a paper title, except for proper nouns and 
element symbols.

For papers published in translation journals, please give the English 
citation first, followed by the original foreign-language citation \cite{b6}.

\begin{thebibliography}{00}
\bibitem{b1} G. Eason, B. Noble, and I. N. Sneddon, ``On certain integrals of Lipschitz-Hankel type involving products of Bessel functions,'' Phil. Trans. Roy. Soc. London, vol. A247, pp. 529--551, April 1955.
\bibitem{b2} J. Clerk Maxwell, A Treatise on Electricity and Magnetism, 3rd ed., vol. 2. Oxford: Clarendon, 1892, pp.68--73.
\bibitem{b3} I. S. Jacobs and C. P. Bean, ``Fine particles, thin films and exchange anisotropy,'' in Magnetism, vol. III, G. T. Rado and H. Suhl, Eds. New York: Academic, 1963, pp. 271--350.
\bibitem{b4} K. Elissa, ``Title of paper if known,'' unpublished.
\bibitem{b5} R. Nicole, ``Title of paper with only first word capitalized,'' J. Name Stand. Abbrev., in press.
\bibitem{b6} Y. Yorozu, M. Hirano, K. Oka, and Y. Tagawa, ``Electron spectroscopy studies on magneto-optical media and plastic substrate interface,'' IEEE Transl. J. Magn. Japan, vol. 2, pp. 740--741, August 1987 [Digests 9th Annual Conf. Magnetics Japan, p. 301, 1982].
\bibitem{b7} M. Young, The Technical Writer's Handbook. Mill Valley, CA: University Science, 1989.
\bibitem{todo} TODO ADD CITATION, Nowhere, 2022
\end{thebibliography}
\vspace{12pt}
\color{red}
IEEE conference templates contain guidance text for composing and formatting conference papers. Please ensure that all template text is removed from your conference paper prior to submission to the conference. Failure to remove the template text from your paper may result in your paper not being published.

\end{document}
